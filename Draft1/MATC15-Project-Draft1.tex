\documentclass{article}
\usepackage[utf8]{inputenc}

\title{Diophantine Equations to the Power of $n$ \\ \vspace{.3in} \large{MATC15 - Project - Draft 1}}
\author{Andrew D'Amario, Kevin ..., ... }
\date{March 2021}

\usepackage{natbib}
\usepackage{graphicx}

\begin{document}

\maketitle

\begin{flushleft}
    {\bf Conjecture:} Let $x$ be an arbitrary integer.

    \hspace{.5in}$\displaystyle x^n=\sum^{n}_{i=1}y_i^n$ has an integer solution such that $y_i\ne x, \forall i$.

    \hspace{2in} Andrew D'Amario, Feburary 18, 2021
\end{flushleft}

\section{Introduction}
The objective of this project is to investigate the conjecture above: whether or not we can always find at least one integer solution to equations of the form $x^n=y_0^n+\cdot\cdot\cdot+y_n^n$ given any $x$, excluding trivial solutions involving $y_i$'s$=0$ or $x$. 

Some of this investigation and research will involve:
\begin{itemize}
    \item Computational analysis on random integers raised to the power of $n$ and finding an integer solution to the sum.
    \item Noting differences between even and odd $n$.
    \item Identifying different families of solutions that take on a similar form.
\end{itemize}

Though this conjecture may be false, we hope to investigate as much as we can on the matter and provide some deeper research to the subject.

\bibliographystyle{plain}
\bibliography{references}
\end{document}
