\documentclass{article}
\usepackage[utf8]{inputenc}

\title{Diophantine Equations to the Power of $n$ \\ \vspace{.3in} \large{MATC15 - Project - Draft 2}}
\author{Andrew D'Amario, Kevin Santos, Dawson Brown}
\date{March 2021}

\usepackage{natbib}
\usepackage{graphicx}
\usepackage{amsmath}
\usepackage{amsfonts}

\begin{document}

\maketitle

\begin{flushleft}
    {\bf Conjecture 1:}

    \hspace{.5in}$\displaystyle x^n=\sum^{n}_{i=1}y_i^n$ has an integer solution such that $y_i\ne x \land y_i > 0, \forall i$.

    \hspace{2in} Andrew D'Amario (A.D.), February 18, 2021

\section{Introduction}
The objective of this project is to investigate the conjecture above: whether or not we can always find at least one integer solution to equations of the form $x^n=y_0^n+\cdot\cdot\cdot+y_n^n$ given any $x$, excluding trivial solutions involving $y_i$'s$=0$ or $x$. This project will be a Type II project. \\

Some of this investigation and research will involve:
\begin{itemize}    
    \item Finding parameters and conditions for possible valid solutions
    \item Computational analysis on random integers raised to the power of $n$ and finding an integer solution to the sum.
    \item Noting differences between even and odd $n$.
    \item Identifying different families of solutions that take on a similar form.
\end{itemize}

Though this conjecture may be false, we hope to investigate as much as we can on the matter and provide some deeper research to the subject.

\section{Searching for Solutions}
Computing these solutions purely by trial and error is prohibitively expensive, even for modern computers. Given some $x \in \mathbb{N}$, if we wish to find a potential solution to the equation in Conjecture 1 (Eqn 1.) naively, we could search all potential combinations of $y_i$ s.t. $y_i \in \{0, 1, 2, ..., x\}$. This, however, results in an algorithm which performs $x^n$ operations in the worst case, which for large values of $x$ - and even moderately sized values of $n$ - is incredibly slow. Thus, before any searching can be done, the potential values of each $y_i$ must be narrowed down.

\vspace{.1in}

We first establish a more resonable upper bound on each $y_i$. Without loss of generality, consider the upper bound for $y_1$. This can be easily extended to any other $y_i$ due to the commutativity of addition, and the fact that they are all raised to the same power. We have: $x^n = \sum^{n}_{i=1}y_i^n$ \\
\hspace{.2in} $\implies x^n = y_1^n + \sum_{i = 2}^ny_i^n$ \\
\hspace{.2in} $\implies y_1^n = x^n - \sum_{i = 2}^ny_i^n$ \\
Note that due to our restrictions, $y_i \geq 1 \forall i$ \\
\hspace{.2in} $\implies y_1^n \leq x^n - (n - 1)$ \\
\hspace{.2in} $\implies y_1 \leq \sqrt[n]{x^n - n + 1}$ \\
While this is indeed less than $x$, for large values of $x$ or $n$ it does not significantly reduce the running time of the algorithm. This means other methods must be employed.

\vspace{.1in}

Another significant reduction comes from the elimination of repeated cases. Due to the commutativity of addition, if we have two cases: $(y_1, \ldots, y_i, \ldots, y_j, \ldots, y_n)$ and $(y_1, \ldots, y_j, \ldots, y_i, \ldots, y_n)$, they will be equivalent, and do not need to be checked twice. Thus, instead of checking $x^n - n + 1$ cases, we need to check a number of cases equivalent to how many ways $\{0, 1, \ldots, \sqrt[n]{x^n - n + 1}\}$ can be uniquely placed in $n$ unordered elements. Employing a common method in statistics, this can be considered a case of `dividers and buckets'. We have $n$ `buckets', and $\sqrt[n]{x^n - n + 1} - 1$ `dividers', which we place between the buckets. Any bucket to the left of the first divider will contain $y_i = 0$, between the first and second divider will be $y_i = 1$, between the second and third will be $y_i = 2$, and so on. Given $\sqrt[n]{x^n - n + 1} - 1 + n$ slots which are sufficient to hold either a divider or a bucket, there are $\sqrt[n]{x^n - n + 1} - 1 + n$ choose $n = \choose{\sqrt[n]{x^n - n + 1} - 1 + n}{n}$ ways to place these elements, which is the new running time of the algorithm. Since $\choose{a}{b} = \frac{a!}{b!(a - b)!}$, the running time of this new algorithm is in the order of $x!$ instead of $x^n$, which is far better.

\vspace{.1in}

One final strategy used by Leech is to examine $x^n mod k$ for some $k$, then eliminate solutions based on those findings (Leech 1958). For example, knowing that $x^4 \mod 16 \in \{0, 1\}$ implies $3$ of the $y_i's$ must be even, while the last must be odd (when the $y_i$'s do not share a common factor), since their sum must be $1 \mod 16$ (Leech, 1958). This makes patterns appearing in powers mod $n$ particularly important to this topic, which lead to some of the proposed patterns in section $4$. One such case of this strategy is discussed in section $3$.

\vspace{.1in}

To assist in finding these reductions, we created an algorithm which checked $x^n \mod k$ for given $x, n$, and all $x$ up to a given $K$. Since $a \equiv b \mod k \implies a^n \equiv b^n \mod k \forall a, b, n, k \in \mathbb{n}$, it was sufficient to check all elements of the reduced residue system of $k$, sort the resulting set, and find the unique elements. Once a good $k$ was found, and restrictions on the $y_i$'s were imposed, we simply ran the naive algorithm and discarded all cases where some $y_i$ did not meet the requirements, usually by changing the `step size' which each $y_i$ was increased by per iteration.

\section{The Case Of Near-Primes}

One of the broadest searches we were able to perform was for $n = 10$. Running the mod-search algorithm described above, we found that $x^{10} \mod 11$ was $0$ or $1$ for all $x \in \mathbb{N}$. Moreover, the only numbers $x$ such that $x^{10} \equiv 0 \mod 11$ were $11k, k \in \mathbb{N}$. Because of this, we know if $x^{10} \equiv 1 \mod 11$, then $\sum^{10}_{i=1}y_i^{10} \equiv 1 \mod 11 \implies$ one $y_i$ is congruent to $1 \mod 11$, and the rest must be congruent to $0 \mod 11$. If $x^10 \equiv 0 \mod 11$, all $y_i$ must be congruent to $0 \mod 11$. Due to the commutativity of addition, we were able to only consider $y_1 \equiv 0$ or $1 \mod 11$, and every other value could be incremented in steps of $11$, allowing us to eliminate a significant number of cases.

\vspace{.1in}

This can be extended to any $n$ s.t. $n = p - 1$ for some prime $p \in \mathbb{n}$. By Fermat's little theorem, given $p$ prime, $\forall x \in \mathbb{N}, p \not| x \implies x^{p - 1} \equiv x^n \equiv 1 \mod p$, and, by the definition of mod, $p | x \implies x \equiv 0 \mod p$. Since $n = p - 1$, we cannot sum enough $y_i$'s such that $y_i \equiv 1 \mod p$ to exceed or equal $p$, so it must be that if $x \equiv 0 \mod p \implies p | x,$ then $\forall i \in \{1, \ldots, n\}, y_1 \equiv 0 \mod p \implies p | y_i$, and if $x \equiv 1 \mod p$, there is only one $y_i$ congruent to $1$ mod $p$. Thus, for further exploration regarding this topic, we recommend searching on values of $n$ that satisfy this condition, as solutions will be found far more easily.

\section{Patterns of Powers$\mod N$}
In order to find solutions for different $n$ we investigated checking the reduced residues if $x^n$ mod $N$ so that we could eliminate solutions that would not lead to a possible solution. 

\vspace{.1in}

While investigating different positive integers $n$ and $N$ we found certain patterns for reduced residues. 

\vspace{.1in}

Let $A$ be the sequence $A = \{0^n,1^n,2^n,3^n,4^n,5^n,...\}$, and $A_r = \{a$ mod $N\}_{a\in A}$ be the reduced residue of $A$ mod $N$ for some $N\in\mathbb{N}$. Here may be some potential conjetures on the set $A$ given the data we have collected:

\vspace{.1in}

{\bf Conjecture 2:} If $n$ is even, every other element starting with the first in $A_r$ is $0$, $A_r=\{0,\_,0,\_,0,\_,...\}$, for $N=2^d$ for all integers $d$. \\
    \hspace{.2in} i.e. $a_{2i}=0$, where $a_{2i}\in A_r$ is the $2i^{th}$ element in $A_r$. \\
    \hspace{4.3in} A.D.

    Collected data for Conjecture 2, $A_r$ for even $n$:
    \begin{itemize}
        \item n = 24, N = 2:  0, 1, 0, 1, 0, 1, 0, 1, 0, 1, 0, 1, 0, 1, 0, 1, ...
            \\ n = 24, N = 4:  0, 1, 0, 1, 0, 1, 0, 1, 0, 1, 0, 1, 0, 1, 0, 1, ...
            \\ n = 24, N = 8:  0, 1, 0, 1, 0, 1, 0, 1, 0, 1, 0, 1, 0, 1, 0, 1, ...
            \\ n = 24, N = 16:  0, 1, 0, 1, 0, 1, 0, 1, 0, 1, 0, 1, 0, 1, 0, 1, ...
            \\ n = 24, N = 32:  0, 1, 0, 1, 0, 1, 0, 1, 0, 1, 0, 1, 0, 1, 0, 1, ...
            \\ n = 24, N = 64:  0, 1, 0, 33, 0, 33, 0, 1, 0, 1, 0, 33, 0, 33, 0, 1, ...
            \\ n = 24, N = 128:  0, 1, 0, 97, 0, 33, 0, 65, 0, 65, 0, 33, 0, 97, 0, 1, ...
            \\ n = 24, N = 256:  0, 1, 0, 225, 0, 161, 0, 65, 0, 193, 0, 33, 0, 97, 0, 129, ...
            \\ n = 24, N = 512:  0, 1, 0, 225, 0, 417, 0, 65, 0, 449, 0, 33, 0, 97, 0, 129, ...
            % \\ n = 24, N = 1024:  0, 1, 0, 225, 0, 417, 0, 65, 0, 449, 0, 545, 0, 97, 0, 641, ...
            % \\ n = 24, N = 8192:  0, 1, 0, 6369, 0, 3489, 0, 5185, 0, 5569, 0, 7713, 0, 7265, 0, 4737, ...
            \\ n = 24, N = 1024:  0, 1, 0, 225, 0, 417, 0, 65, 0, 449, 0, 545, 0, 97, 0, ...
            \\ n = 24, N = 8192:  0, 1, 0, 6369, 0, 3489, 0, 5185, 0, 5569, 0, 7713, 0, ...
        \item n = 10, N = 32:  0, 1, 0, 9, 0, 25, 0, 17, 0, 17, 0, 25, 0, 9, 0, 1, ...
        \item n = 12, N = 64:  0, 1, 0, 49, 0, 17, 0, 33, 0, 33, 0, 17, 0, 49, 0, 1, ...
        \item n = 14, N = 512:  0, 1, 0, 377, 0, 489, 0, 81, 0, 305, 0, 137, 0, 473, 0, 33, ...
        \item n = 18, N = 2:  0, 1, 0, 1, 0, 1, 0, 1, 0, 1, 0, 1, 0, 1, 0, 1, ...
        \item n = 34, N = 128:  0, 1, 0, 9, 0, 25, 0, 49, 0, 81, 0, 121, 0, 41, 0, 97, ...
        \item n = 235676, N = 128:  0, 1, 0, 49, 0, 17, 0, 33, 0, 97, 0, 81, 0, 113, 0, 65, ...
    \end{itemize}
    
\vspace{.1in}

{\bf Conjecture 2.1:} If $n=2^k$ for some integer $k$, there exists a natural number $N$ such that $A_r$ is of the form $\{0,1,0,1,0,1,...\}$. \\
Moreover, for $k>1$, $A_r$ has this form for all $N=2^d$, $d\in\{1,...,k+2\}$. \\
    \hspace{4.3in} A.D.

    Collected data for Conjecture 2.1, $A_r$ for $n=2^k$:
    \begin{itemize}
        \item n = 1, N = 2:  0, 1, 0, 1, 0, 1, 0, 1, 0, 1, 0, 1, 0, 1, 0, 1, ...
        \item n = 2, N = 2:  0, 1, 0, 1, 0, 1, 0, 1, 0, 1, 0, 1, 0, 1, 0, 1, ...
            \\ n = 2, N = 4:  0, 1, 0, 1, 0, 1, 0, 1, 0, 1, 0, 1, 0, 1, 0, 1, ...
        \item n = 4, N = 2:  0, 1, 0, 1, 0, 1, 0, 1, 0, 1, 0, 1, 0, 1, 0, 1, ...
            \\ n = 4, N = 4:  0, 1, 0, 1, 0, 1, 0, 1, 0, 1, 0, 1, 0, 1, 0, 1, ...
            \\ n = 4, N = 8:  0, 1, 0, 1, 0, 1, 0, 1, 0, 1, 0, 1, 0, 1, 0, 1, ...
            \\ n = 4, N = 16:  0, 1, 0, 1, 0, 1, 0, 1, 0, 1, 0, 1, 0, 1, 0, 1, ...
        \item n = 8, N = 2:  0, 1, 0, 1, 0, 1, 0, 1, 0, 1, 0, 1, 0, 1, 0, 1, ...
            \\ n = 8, N = 4:  0, 1, 0, 1, 0, 1, 0, 1, 0, 1, 0, 1, 0, 1, 0, 1, ...
            \\ n = 8, N = 8:  0, 1, 0, 1, 0, 1, 0, 1, 0, 1, 0, 1, 0, 1, 0, 1, ...
            \\ n = 8, N = 16:  0, 1, 0, 1, 0, 1, 0, 1, 0, 1, 0, 1, 0, 1, 0, 1, ...
            \\ n = 8, N = 32:  0, 1, 0, 1, 0, 1, 0, 1, 0, 1, 0, 1, 0, 1, 0, 1, ...
        \item n = 16, N = 2:  0, 1, 0, 1, 0, 1, 0, 1, 0, 1, 0, 1, 0, 1, 0, 1, ...
            \\ n = 16, N = 4:  0, 1, 0, 1, 0, 1, 0, 1, 0, 1, 0, 1, 0, 1, 0, 1, ...
            \\ n = 16, N = 8:  0, 1, 0, 1, 0, 1, 0, 1, 0, 1, 0, 1, 0, 1, 0, 1, ...
            \\ n = 16, N = 16:  0, 1, 0, 1, 0, 1, 0, 1, 0, 1, 0, 1, 0, 1, 0, 1, ...
            \\ n = 16, N = 32:  0, 1, 0, 1, 0, 1, 0, 1, 0, 1, 0, 1, 0, 1, 0, 1, ...
            \\ n = 16, N = 64:  0, 1, 0, 1, 0, 1, 0, 1, 0, 1, 0, 1, 0, 1, 0, 1, ...
        \item n = 32, N = 2:  0, 1, 0, 1, 0, 1, 0, 1, 0, 1, 0, 1, 0, 1, 0, 1, ...
            \\ n = 32, N = 4:  0, 1, 0, 1, 0, 1, 0, 1, 0, 1, 0, 1, 0, 1, 0, 1, ...
            \\ n = 32, N = 8:  0, 1, 0, 1, 0, 1, 0, 1, 0, 1, 0, 1, 0, 1, 0, 1, ...
            \\ n = 32, N = 16:  0, 1, 0, 1, 0, 1, 0, 1, 0, 1, 0, 1, 0, 1, 0, 1, ...
            \\ n = 32, N = 32:  0, 1, 0, 1, 0, 1, 0, 1, 0, 1, 0, 1, 0, 1, 0, 1, ...
            \\ n = 32, N = 64:  0, 1, 0, 1, 0, 1, 0, 1, 0, 1, 0, 1, 0, 1, 0, 1, ...
            \\ n = 32, N = 128:  0, 1, 0, 1, 0, 1, 0, 1, 0, 1, 0, 1, 0, 1, 0, 1, ...
        \item n = 64, N = 2:  0, 1, 0, 1, 0, 1, 0, 1, 0, 1, 0, 1, 0, 1, 0, 1, ...
            \\ n = 64, N = 4:  0, 1, 0, 1, 0, 1, 0, 1, 0, 1, 0, 1, 0, 1, 0, 1, ...
            \\ n = 64, N = 8:  0, 1, 0, 1, 0, 1, 0, 1, 0, 1, 0, 1, 0, 1, 0, 1, ...
            \\ n = 64, N = 16:  0, 1, 0, 1, 0, 1, 0, 1, 0, 1, 0, 1, 0, 1, 0, 1, ...
            \\ n = 64, N = 32:  0, 1, 0, 1, 0, 1, 0, 1, 0, 1, 0, 1, 0, 1, 0, 1, ...
            \\ n = 64, N = 64:  0, 1, 0, 1, 0, 1, 0, 1, 0, 1, 0, 1, 0, 1, 0, 1, ...
            \\ n = 64, N = 128:  0, 1, 0, 1, 0, 1, 0, 1, 0, 1, 0, 1, 0, 1, 0, 1, ...
            \\ n = 64, N = 256:  0, 1, 0, 1, 0, 1, 0, 1, 0, 1, 0, 1, 0, 1, 0, 1, ...
    \end{itemize}

\vspace{.1in}

{\bf Conjecture 3:} If $n$ is prime, there exists a natural number $N$ such that $A_r$ is of the form: $A_r=\{0,1,2,3,...,n-1,0,1,2,3,...,n-1,...\}$. \\
        \hspace{4.3in} A.D.
        
        Collected data for Conjecture 3, $A_r$ for prime $n$:
        \begin{itemize}
            \item n = 3, N = 3:  0, 1, 2, 0, 1, 2, 0, 1, 2, 0, 1, 2, 0, 1, 2, 0, ...
            \item n = 5, N = 5:  0, 1, 2, 3, 4, 0, 1, 2, 3, 4, 0, 1, 2, 3, 4, 0, ...
            \item n = 7, N = 7:  0, 1, 2, 3, 4, 5, 6, 0, 1, 2, 3, 4, 5, 6, 0, 1, 2, 3, 4, 5, 6, ...
            \item n = 11, N = 11:  0, 1, 2, 3, 4, 5, 6, 7, 8, 9, 10, 0, 1, 2, 3, 4, 5, 6, 7, 8, 9, 10, 0, 1, 2, 3, 4, 5, 6, 7, 8, 9, 10, 0, 1, 2, ...
        \end{itemize}

\newpage

\bibliographystyle{plain}
\bibliography{references}
\section{References}
\begin{itemize}
    \item Drago Bajc, {\bf Power solutions of some Diophantine equations}, \\
        \textit{The Mathematical Gazette}, 97:538, 107-110 (2013). \\
        https://www.jstor.org/stable/24496765
        \subitem Mentions form of above conjecture and states that solutions have been found in some cases but not in other cases, such as $n=6$. Considers above conjecture with $x^k$ instead of $x^n$, where $(k,n)=1$ and provides a general form for these solutions. 
    \item {\bf Computing Minimal Equal Sums Of Like Powers}, \\
        http://euler.free.fr/index.htm 
        \subitem Website dedicated to finding and compiling examples and counterexamples of Euler's sums of powers conjecture, which states that if a sum of $n$ positive $kth$ powers equals one $kth$ power, then $n>=k$. Includes many resources we can look into. 
    \item {\bf BEST KNOWN SOLUTIONS}, \\
        http://euler.free.fr/records.htm
        \subitem  Extensive list of aforementioned examples and counterexamples to Euler's sums of powers conjecture.
    \item L. Jacobi, D. Madden, {\bf On $a^4 + b^4 + c^4 + d^4=(a+b+c+d)^4$}, \\
        \textit{The American Mathematical Monthly}, 115:3, 230-236 (2008). \\
        https://doi.org/10.1080/00029890.2008.11920519
        \subitem Discusses specific case of the conjecture with $n=4$. Also discusses relation of Euler's conjecture and related Diophantine equations to the topic of elliptic curves. 
    \item T. Roy and F. J. Sonia, {\bf A Direct Method To Generate Pythagorean Triples And Its Generalization To Pythagorean Quadruples And n-tuples},\\ 
    https://arxiv.org/ftp/arxiv/papers/1201/1201.2145.pdf 
        \subitem Gives methods for finding Pythagorean n-tuples, sums of $n$ squares that result in a square. Might be able to reduce some cases into one of these cases. 
    \item D. R. Heath-Brown, W. M. Lioen and H. J. J. Te Riele, {\bf On Solving the Diophantine Equation $x^3 + y^3 + z^3 = k$ on a Vector Computer},\\
        \textit{Mathematics of Computation}, 61:203, 235-244 (1993)
        \subitem Presents detailed algorithm for the $n=3$ case, might be able to apply similar principles with higher $n$ values. \\
    \item L. J. Lander, T. R. Parkin and J. L. Selfridge,
       {\bf A survey of equal sums of like powers}, \\
        \textit{Mathematics of Computation}, 21, 446-459 (1967). \\
        https://www.ams.org/journals/mcom/1967-21-099/S0025-5718-1967-0222008-0/S0025-5718-1967-0222008-0.pdf 
        \subitem Presents various solutions to powers of Diophantine equations, \\
        including the $n=4$ and $n=5$ cases of the conjecture. 
    \item J. Leech, {\bf On $A^4 + B^4 + C^4 + D^4 = E^4$}, \\ 
        \textit{Mathematical Proceedings of the Cambridge Philosophical Society}, 54(4), 554-555, (1958). \\
        doi.org/10.1017/S0305004100003091
        \subitem Brief paper outlining found solutions for the $n=4$ case and considerations that reduce the number of possible solutions that need to be checked. 
\end{itemize}

\end{flushleft}

\end{document}