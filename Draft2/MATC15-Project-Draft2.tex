\documentclass{article}
\usepackage[utf8]{inputenc}

\title{Diophantine Equations to the Power of $n$ \\ \vspace{.3in} \large{MATC15 - Project - Draft 2}}
\author{Andrew D'Amario, Kevin Santos, Dawson Brown }
\date{March 2021}

\usepackage{natbib}
\usepackage{graphicx}

\begin{document}

\maketitle

\begin{flushleft}
    {\bf Conjecture:}

    \hspace{.5in}$\displaystyle x^n=\sum^{n}_{i=1}y_i^n$ has an integer solution such that $y_i\ne x, \forall i$.

    \hspace{2in} Andrew D'Amario, February 18, 2021
\end{flushleft}

\section{Introduction}
The objective of this project is to investigate the conjecture above: whether or not we can always find at least one integer solution to equations of the form $x^n=y_0^n+\cdot\cdot\cdot+y_n^n$ given any $x$, excluding trivial solutions involving $y_i$'s$=0$ or $x$. This project will be a Type II project. \\

Some of this investigation and research will involve:
\begin{itemize}    
    \item Finding parameters and conditions for possible valid solutions
    \item Computational analysis on random integers raised to the power of $n$ and finding an integer solution to the sum.
    \item Noting differences between even and odd $n$.
    \item Identifying different families of solutions that take on a similar form.
\end{itemize}

Though this conjecture may be false, we hope to investigate as much as we can on the matter and provide some deeper research to the subject.

\bibliographystyle{plain}
\bibliography{references}
\section{References}
\begin{itemize}
    \item Drago Bajc, {\bf Power solutions of some Diophantine equations}, \\
        \textit{The Mathematical Gazette}, 97:538, 107-110 (2013). \\
        https://www.jstor.org/stable/24496765
        \subitem Mentions form of above conjecture and states that solutions have been found in some cases but not in other cases, such as $n=6$. Considers above conjecture with $x^k$ instead of $x^n$, where $(k,n)=1$ and provides a general form for these solutions. 
    \item {\bf Computing Minimal Equal Sums Of Like Powers}, \\
        http://euler.free.fr/index.htm 
        \subitem Website dedicated to finding and compiling examples and counterexamples of Euler's sums of powers conjecture, which states that if a sum of $n$ positive $kth$ powers equals one $kth$ power, then $n>=k$. Includes many resources we can look into. 
    \item {\bf BEST KNOWN SOLUTIONS}, \\
        http://euler.free.fr/records.htm
        \subitem  Extensive list of aforementioned examples and counterexamples to Euler's sums of powers conjecture.
    \item L. Jacobi, D. Madden, {\bf On $a^4 + b^4 + c^4 + d^4=(a+b+c+d)^4$}, \\
        \textit{The American Mathematical Monthly}, 115:3, 230-236 (2008). \\
        https://doi.org/10.1080/00029890.2008.11920519
        \subitem Discusses specific case of the conjecture with $n=4$. Also discusses relation of Euler's conjecture and related Diophantine equations to the topic of elliptic curves. 
    \item T. Roy and F. J. Sonia, {\bf A Direct Method To Generate Pythagorean Triples And Its Generalization To Pythagorean Quadruples And n-tuples},\\ 
    https://arxiv.org/ftp/arxiv/papers/1201/1201.2145.pdf 
        \subitem Gives methods for finding Pythagorean n-tuples, sums of $n$ squares that result in a square. Might be able to reduce some cases into one of these cases. 
    \item D. R. Heath-Brown, W. M. Lioen and H. J. J. Te Riele, {\bf On Solving the Diophantine Equation $x^3 + y^3 + z^3 = k$ on a Vector Computer},\\
        \textit{Mathematics of Computation}, 61:203, 235-244 (1993)
        \subitem Presents detailed algorithm for the $n=3$ case, might be able to apply similar principles with higher $n$ values. \\
    \item L. J. Lander, T. R. Parkin and J. L. Selfridge,
       {\bf A survey of equal sums of like powers}, \\
        \textit{Mathematics of Computation}, 21, 446-459 (1967). \\
        https://www.ams.org/journals/mcom/1967-21-099/S0025-5718-1967-0222008-0/S0025-5718-1967-0222008-0.pdf 
        \subitem Presents various solutions to powers of Diophantine equations, \\
        including the $n=4$ and $n=5$ cases of the conjecture. 
    \item J. Leech, {\bf On $A^4 + B^4 + C^4 + D^4 = E^4$}, \\ 
        \textit{Mathematical Proceedings of the Cambridge Philosophical Society}, 54(4), 554-555, (1958). \\
        doi.org/10.1017/S0305004100003091
        \subitem Brief paper outlining found solutions for the $n=4$ case and considerations that reduce the number of possible solutions that need to be checked. 
\end{itemize}

\end{document}